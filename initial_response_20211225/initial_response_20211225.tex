% Copyright Javier Sánchez-Monedero.
% Please report bugs and suggestions to (jsanchezm at uco.es)
%
% This document is released under a Creative Commons Licence 
% CC-BY-SA (http://creativecommons.org/licenses/by-sa/3.0/) 
%
% BASIC INSTRUCTIONS: 
% 1. Load and set up proper language packages
% 2. Complete the paper data commands
% 3. Use commands \rcomment and \newtext as shown in the example



%----------------------------------------------------------------
%            content
%----------------------------------------------------------------
%
%Integrating arara so that we can compile package multibib under 
%windows; see 
%https://tex.stackexchange.com/questions/183858/how-to-configure-texstudio-editor-for-multibib
%https://tex.stackexchange.com/questions/100318/how-to-install-arara-with-miktex-windows
%https://tex.stackexchange.com/questions/448129/arara-automation-tool-script-not-found
% arara: pdflatex: {synctex: yes}
% !arara: makeindex
% arara: biber
% arara: pdflatex: {synctex: yes}
% arara: pdflatex: {synctex: yes}

\documentclass[a4paper,twoside,11pt]{reviewresponse}
% 1. Load and set up proper language packages
%\usepackage[utf8x]{inputenc}
\usepackage[utf8]{inputenc}
\usepackage[T1]{fontenc}
\usepackage[english]{babel}
\usepackage{xspace}

\usepackage[defaultlines=4,all]{nowidow}%forbid widow or orphan
% https://tex.stackexchange.com/questions/229638/package-biblatex-warning-babel-polyglossia-detected-but-csquotes-missing
\usepackage{csquotes}

% Hack to try to make acmart work with biblatex: 
%https://tex.stackexchange.com/questions/37076/is-it-possible-to-load-biblatex-with-a-class-that-has-already-loaded-natbib
%\let\citename\relax
\usepackage[
backend=bibtex, % works on Overleaf
% backend=biber, 
%style=numeric-comp, %number the references, compact
style=authoryear, %no indices before names; year after names
%style=alphabetic, %style for my thesis
%
sorting=nyvt, %sort in bibliography by name, year, volume, title
%sortcites, %sort in content by numbering
hyperref=true,%transform citations and back references into 
%clickable hyperlinks
backref=true, %to print back references in the bibliography
backrefstyle=three, %Compress three or more consecutive pages 
%to a range
%,
%citestyle=authoryear,
maxbibnames=9, %the author names appearing in bibliography
maxcitenames=2, %the author names appearing in text
giveninits=true,
uniquename=init, % see tex.stackexchange.com/questions/505654/
%terseinits=true,%remove periods after initials of first names
isbn=false,
%doi=false,url=false,
eprint=false,
%dashed=false,
%useprefix=false,%sort bibliography use prefix like "van"
date=year, %so that the month will not shown with the year
natbib=true,
%style=ACM-Reference-Format,
language=american,
abbreviate=true, dateabbrev=true,
]{biblatex}

%\addbibresource{../Reference/BibReference.bib} % works on local computer
\addbibresource{Reference/BibReference.bib} % works on Overleaf
% \addbibresource{BibReference.bib} % The copy in this folder
\DeclareNameAlias{author}{family-given}%lastname before firstname


%if a reference has doi, then we print only doi in bibliography;
%otherwise, we print only url
%we don't use eprint
%we must deactivate eprint by using "eprint=false"
\renewbibmacro*{doi+eprint+url}{% 
	\iftoggle{bbx:url} 
	{\iffieldundef{doi}{\usebibmacro{url+urldate}}{}} 
	{}% 
	%	\newunit\newblock 
	%	\iftoggle{bbx:eprint} 
	%	{\usebibmacro{eprint}} 
	%	{}% 
	\newunit\newblock 
	\iftoggle{bbx:doi} 
	{\printfield{doi}} 
	{}}	


\newcommand{\Astar}{A$^{\!\star}$\xspace}

% 2. Complete the paper data
\newcommand{\myAuthors}{
{Dongliang Peng, 
Martijn Meijers, 
Peter van Oosterom} %\\ {} other authors
}


\newcommand{\myAuthorsShort}{Dongliang Peng et. al.}
\newcommand{\myEmail}{}
\newcommand{\myTitle}
{Initial responses to the reviewers of paper 
``Generalizing simultaneously to support smooth zooming:
Case study of merging area objects''}
\newcommand{\myShortTitle}{Finding Optimal Sequences for Area Aggregation}
\newcommand{\myJournal}{International Journal of Geographical Information Science}
\newcommand{\myDept}{
    {GIS Technology, Delft University of Technology, The Netherlands}}
%%%%%%%%%%%%%%%%%%%%%%%%%%%%%%%%%%%%%%%%%%%%%%%%%%%%%%%%%%%%%%%%%%%%%%%%%%



%\usepackage[linktoc=all]{hyperref}
\usepackage[linktoc=all,bookmarks,
bookmarksopen=true,bookmarksnumbered=true]{hyperref}

\hypersetup{
	pdfauthor = {\myAuthorsShort},
	pdftitle = {\myTitle},
	pdfsubject = {\myJournal\xspace},
	colorlinks = true,
	linkcolor=black!70!green,          % color of internal links
	citecolor=black!70!green,        % color of links to bibliography
	filecolor=magenta,      % color of file links
	urlcolor=black!70!green           % color of external links
}


\newcommand{\e}[1]{\times 10^{#1}}
\newcommand{\fig}{Figure~}
\newcommand{\eq}{Equation~}
\newcommand{\fo}{Formula~}
\newcommand{\sect}{Section~}
\newcommand{\tab}{Table~}
\newcommand{\chap}{Chapter~}
\newcommand{\figs}{Figures~}
\newcommand{\eqs}{Equations~}
\newcommand{\fos}{Formulas~}
\newcommand{\sects}{Sections~}
\newcommand{\tabs}{Tables~}
\newcommand{\chaps}{Chapters~}
\newcommand{\p}{p.~}
\newcommand{\pp}{pp.~}
\newcommand{\eg}{e.g.,}
\newcommand{\ie}{i.e.,}

\renewcommand{\labelitemi}{\textendash}




\begin{document}
	
\thispagestyle{plain}

\begin{center}
	{\Large\myTitle} \vspace{0.5cm} \\
%	{\large\myJournal} \vspace{0.5cm} \\
	\today \vspace{0.5cm} \\
%	\myAuthors \\
%	\url{\myEmail} \vspace{1cm} \\
%	\myDept
\end{center}

\tableofcontents

\begin{abstract}
%We submitted an earlier version of this paper to 
%\emph{International Journal of Geographical Information Science}
%on May 27, 2020.
%According to the criticisms of the reviewers,
%the Associate Editor, Professor Bo Huang, rejected that version.
%However, Bo would welcome a new submission with substantial reworking 
%of that version taking into consideration the reviewers' comments. 
%
%We thank all the anonymous reviewers for their comments.
%These comments were very helpful for us to improve our paper. 
%In this document, 
%we provide detailed responses to each of the reviewers' comments. 
%The following list summarizes our most important changes:
%\begin{itemize}
%    \item  change 1.
%    \item  change 2.
%\end{itemize}
\end{abstract}

\section{Reviewer 1}
\setcounter{comments}{-1} %the counter will start from 0

\rcommentnoskip{
This paper proposes a parallel merging method based on a greedy algorithm and a space-scale cube
(SSC) to generalize a digital map continuously and smoothly. According to cases and comparisons
accessed at webmaps\footnote{%
The case study at
\url{https://congengis.github.io/webmaps/2020/05/merge/example-discrete-merging/}.}, the parallel merging method can provide users with gradual impression.
Although the proposed method seems interesting and valid, some points should be carefully
considered to publish this paper.
}

\citep[\eg][]{Noellenburg2008,Li2017Annealing}

%\textbf{Response.} Thanks for the summary.
%
%
%\rcomment{
%1. The content of Abstract (Line 1--32) cannot reflect the highlights of this paper, 
%which is focused too much on the implement of the proposed method.
%}
%
%\textbf{Response.} We have improved our abstract. 
%The new abstract focuses more on the benefits and novelty of our method.
%
%
%\rcomment{
%2. There have been many continuous map generalization algorithms for different features to provide
%scale transition. And authors mainly introduce algorithms about area object. Except for the merging
%operation to generalize area objects, if there are others existing parallel methods to generalize
%them? (Line 6--29)
%}
%
%\textbf{Response.} We have added the following paragraph.
%
%Many methods of CMG naturally parallel generalization operators.
%In morphing polylines, the points of the polylines are moved parallelly
%\citep[\eg][]{Noellenburg2008,Li2017Annealing}.
%\citet{Li2017_Building} parallelly generalized individual buildings.
%In those methods, the polylines and the buildings 
%were generalized parallelly 
%because they were handled independently.
%\citet{Peng2017Building,Touya2017Progressive}
%generalized building to built-up areas.
%However, there is no simple relationship 
%between their intermediate-scale maps and their source maps.
%Therefore, all the intermediate-scale maps of buildings have to be
%sent from the server to the clients,
%which is network intensive.
%
%
%\rcomment{
%3. The authors introduce researches studied by \textcite{Suba2016Road}
%and \textcite{Huang2017Matrix}, however, 
%the differences of these methods compared with 
%the method of \textcite{vanOosterom2014Support} 
%which is used in this paper are not summarized.
%}
%
%\textbf{Response.} 
%To emphasize the difference between 
%\textcite{Suba2016Road} and \textcite{vanOosterom2014Support},
%we added statement 
%``while \citet{Suba2016Road} approached CMG 
%with small sudden changes,
%\citet{vanOosterom2014Support} provided real smooth changes.''
%
%The research of \textcite{Huang2017Matrix} 
%is about continuous map generalization of river networks,
%which is not directly related to our research of area objects.
%Therefore, we have removed the details of \textcite{Huang2017Matrix}
%and moved the reference into the introduction.
%
%
%\rcomment{
%4. In Figure 2 and Section 2--3, the authors introduce the topological Generalized Area Partitioning
%(tGAP) tree and space-scale cube (SSC), which are very important for the proposed method.
%However, it seems that the SSC built for parallel merging only changes some calculation equation
%compared with the SSC for single merging, which is not enough to highlight the parallel method.
%}
%
%\textbf{Response.}
%The main contribution of this paper is 
%to parallel generalization operators,
%which is the first time that the parallelization 
%is explicitly proposed.
%The parallel merging operations must be well distributed,
%which pushes the boundaries of tGAP and the SSC.
%Because of the parallelization, 
%we need to snap to the scales with completed generalization operations
%and need to discuss the animation time gained for each generalization operation.
%
%We have added the above argument into the introduction.
%
%\rcomment{
%5. If the animation duration of step (in Section 3.5) should be calculated by some initial values or
%empirical value, such as zooming/panning duration?
%}
%
%\textbf{Response.}
%As no research has recommended values for 
%the zooming factor or the zooming duration,
%we respectively set the default values to $1$ and $1\,$s, 
%which performed well according to our experience.
%
%We have added the above argument into
%the section of Animation duration of a step.
%
%\rcomment{
%6. In Section 4, the authors use only one topographic map 
%with $13{,}238$ parcels to implement experiments. 
%This dataset can prove the validity of the proposed method, 
%however, it is not sure whether 
%this method is useful or efficient 
%for topographic map with more parcels.
%}
%
%\textbf{Response.} For a topographic map with much more objects,
%we will need to make $3$d tiles of the map data.
%Then, the client side will dynamically load and process the map data 
%for the place and the scale being viewed.
%The client also needs to remove the loaded data that is not used for a while 
%in order to release memory.
%With those functionalities, our prototype will be able to handle 
%a map with arbitrary number of area objects.
%Those functionalities are under development 
%and will be presented in another paper.
%
%We have added the above discussion into our future work.
%
%
%\rcomment{
%7. The other important problem of case study in Section~4, 
%is that the comparison of $r_\mathrm{parallel}$ is not enough. 
%The authors analysis experimental results according to Table~2 and 
%only give an optimal parameter (0.01). 
%If 0.01 is optimal for all datasets which is not used in this paper?
%}
%
%\textbf{Response.} Parallel parameter~$0.01$ 
%is a safe choice for any kind of dataset,
%but we are far from claiming that value~$0.01$ is optimal
%for all the kinds of datasets.
%Finding optimal parallel parameters for 
%different kinds of datasets are out the scope of this paper.
%
%We have added the discussion into our future work.
%
%
%\rcomment{
%8. The authors conclude this paper (Section 5) 
%in detail which seems to be length. 
%Moreover, the future work (Section 5.2) points out 
%many problems of this proposed method, 
%such as the definition of merging steps, 
%the distribution of merging events and 
%the simplification of boundaries. 
%These issues imply the insufficiency of this work. 
%Therefore, further development of this parallel operation
%method is recommend.
%}
%
%\textbf{Response.}
%
%
%\section{Reviewer 2}
%\setcounter{comments}{-1} %the counter will start from 0
%
%\rcommentnoskip{
%The manuscript presents a method to support a better understanding of maps, and the authors also
%developed an application that is very helpful for readers to get the “parallel” idea. This manuscript
%was organized in a clear way and written well.
%}
%
%\textbf{Response.} 
%
%
%\rcomment{
%The proposed method aims to improve user-friendly when zoom in and out a map, however, my
%major concern also lies in the practical usage of this method. It seems that the authors try to avoid
%the sense of “level” when a user zooming in and out a map, yet the advantage of using LOD (i.e.
%fixed levels) was not discussed and compared with the proposed method. Using the pre-generated
%tiles (image tile and vector tiles) can largely reduce the data transmission amount which can also
%improve the experience of using a map (not only for web-based map applications but also for clientbased
%map applications). In the demonstrative system, the most detailed data need to be loaded
%before presenting the map. Such a way may not suitable for the general public as it requires waiting
%a long time to see the map.
%}
%
%\textbf{Response.}
%
%
%
%
%
%\rcomment{
%Secondly, the memory cost of this method is also my concern. If map data in level-17 need to be
%pre-loaded when a user just pan the map in level-10, the memory cost will apparently exceed. Is
%the proposed method work in this manner? Is that possible to give some more detailed explanation
%about the memory cost?
%}
%
%\textbf{Response.} In our case study, all the data of the SSC is store in a single file
%because the tested map is not very big.
%The map will display only after the whole file is loaded.
%For a topographic map with much more objects,
%We are also developing a method that divides the SSC into many parts, 
%and each part is stored in a file.
%A file will be dynamically loaded when the user is reading the relevant place and scale of the map.
%Further, a file at the client side will be removed to release main memory
%if the corresponding part of map is not browsed for a long time. 
%With those functionalities, our prototype will be able to handle 
%a map with arbitrary number of area objects.
%Those functionalities are under development 
%and will be presented in another paper in future.
%
%We have added the above discussion in our future work.
%
%
%\rcomment{
%Thirdly, if the proposed method is not used in web-based scenarios, the multi-threading strategy
%can be considered to improve efficiency. As web browsers (specifically, the JavaScript language) do
%not support multi-threading, it is OK that merger different events into a step. It would be much
%better if the authors illustrate the “big picture” more clearly.
%}
%
%\textbf{Response.}
%
%\rcomment{
%With all due respect, section 3.5 is a little bit simple and makes the article looks like a technical
%report. The authors can also put more attention on the comparison with some other web-based map
%applications which generally not employ animation.
%}
%
%\textbf{Response.}
%
%
%
%\section{Reviewer 3}
%\setcounter{comments}{-1} %the counter will start from 0
%
%
%\rcommentnoskip{
%This manuscript tries to explore the question of smooth zooming by paralleling generalization
%operations. The idea is based on well known generalization model, GAP model, which is used to
%process area feature generalization using the principle one parcel “eating” a neighbor. The
%manuscript presents an innovation method in the merge integration to act as animation considering
%interested focusing area. As each step just “eating” one unimportant parcel is a time-costing
%process, the study distinguishes the interested area and the others, and for interested area, the
%merge process can be carried out by detailed change using the presented control way. However, the
%following questions need to care.
%}
%
%\textbf{Response.} Thanks for the summary.
%
%We are sorry for the misunderstanding.
%Different users may be interested into different area objects.
%Therefore, they may have different interested areas.
%However, our animation is independent of users' interested areas.
%The merging is always realized by eating
%(no matter the merging operation happens in the interested area).
%We have made this more clearly
%at the beginning of the section of methodology.
%
%
%\rcomment{
%1) The whole manuscript has a big question focusing on your own concrete point without introducing
%the background in a general way from the perspective of map generalization or area feature
%generalization. In the introduction, you used a lot of texts to explore your own point by very
%detailed steps with figure 1 and figure 2. I expect to know the general introduction of your research
%background. The references are also mainly from GAP based. However the other area feature
%generalization need to discuss from a high level of perspective. I think in the introduction you
%should at least consider the following : area feature aggregation, continuous generalization,
%semantic neighbor process, parallel computations and others. I strongly suggest you move two
%figures to section 3 method. Actually in current section 3, you continues to explain figure 1.
%}
%
%\textbf{Response.} 
%
%We have moved Figures~1 and~2 to Section~3.
%
%\rcomment{
%2) For the figure caption, 
%you use a big paragraph with too many texts to describe meaning. 
%I suggest to arrange the main description in the main text body and 
%the caption should be simple and abstract the key ideas.
%}
%
%\textbf{Response.} We have removed some texts 
%in the captions of Figures~1, 2, and~4.
%We have moved some texts of those figures into the main text body.
%
%\rcomment{
%3) For the related works, the first sentence appears by your own reference followed by a list of other
%papers without a background discussion. The related works mainly covers GAP, tGAP, SSC and other
%topics, just within your team works. The perspective level should go up and consider some related
%points in question 1 I mentioned.
%}
%
%\textbf{Response.}
%
%
%
%\rcomment{
%4) In the introduction and related works, 
%you mentioned aggregation operation for land-use generation, 
%another similar operation “amalgamation” needs to compare. 
%If small parcels distribute
%at a large region with very small gap distance to each other, 
%the background parcel will eat all small parcels. 
%Actually combining all neighboring small parcels together 
%to be a big one is more reasonable. 
%The question exists in the neighborhood definition. 
%Those with small gap distance to
%each other need also to be assigned to neighbors. 
%The combination of this situation other than touch
%relationship belongs to amalgamation. 
%The early GAP model is short to process the neighbor with
%small spatial distance, just considers the touching topological neighbor.
%}
%
%\textbf{Response.} 
%We have reviewed some papers about ``amalgamation'' 
%in the section of related work.
%
%If some small parcels distribute at a large region,
%then the background parcel will eat all the small parcels.
%This case already happens in our web maps
%when some small buildings are inside a settlement area.
%In order to solve this problem, 
%we could generate a tessellation of the small parcels
%using the method of \citet{Ai2015Building}.
%Then, we could grow each of the small parcels 
%inside its own cell.
%By then, the small parcels will touch each other
%and can be merged by eating each other.
%
%We have added the above idea into our future work.
%
%
%
%
%
%
%\rcomment{
%5) In the method section, as for the duration costs, the measure method should present. The parcel
%size, number and distribution density pattern can be considered to compute the merge speed in the
%animation process.
%}
%
%\textbf{Response.}
%
%
%\rcomment{
%6) The block process is a key question and depends on how interested area defines. To define
%interested area, do you select certain parcel with some class type, some geometric characteristics
%(big size), distribution in certain region or others ? If within one region, they are neighbors to each
%other, how to process block?
%}
%
%\textbf{Response.}
%
%
%\rcomment{
%7) Could you give an practical example using the same parcel data 
%but different interested areas?
%And make the comparison between two animation processes.
%}
%
%\textbf{Response.} 
%We are sorry for the misunderstanding.
%Different users may be interested in different area objects, 
%and our animation is independent of users' interested areas.
%Therefore, it is impossible to make a comparison 
%between two animation processes 
%because the animation process is always the same.
%
%We have made this more clearly
%at the beginning of the section of methodology.
%
%
%\rcomment{
%8) As for the animation process, suggest to improve the figures using some frames to illustrate the
%progressive change rather than a final result.
%}
%
%\textbf{Response.}
%
%
%\rcomment{
%For figure 10, using the progressive change of color brightness is a very easy job, why let it to be
%future works. In the conclusions, too many future works exist. As for the skeleton process in GAP
%model, the paper ``GAP tree extensions based on skeletons'' has settled. You can reference it.
%}
%
%\textbf{Response.}
%We have realized the progressive change of color brightness.
%In the section of future work, we have referenced paper 
%``GAP-tree extensions based on skeletons'' \citep{Ai2002GAP}. 





\printbibliography
	
\end{document}

