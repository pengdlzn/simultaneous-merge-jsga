% Copyright Javier Sánchez-Monedero.
% Please report bugs and suggestions to (jsanchezm at uco.es)
%
% This document is released under a Creative Commons Licence 
% CC-BY-SA (http://creativecommons.org/licenses/by-sa/3.0/) 
%
% BASIC INSTRUCTIONS: 
% 1. Load and set up proper language packages
% 2. Complete the paper data commands
% 3. Use commands \rcomment and \newtext as shown in the example



%----------------------------------------------------------------
%            content
%----------------------------------------------------------------
%
%Integrating arara so that we can compile package multibib under 
%windows; see 
%https://tex.stackexchange.com/questions/183858/how-to-configure-texstudio-editor-for-multibib
%https://tex.stackexchange.com/questions/100318/how-to-install-arara-with-miktex-windows
%https://tex.stackexchange.com/questions/448129/arara-automation-tool-script-not-found
% arara: pdflatex: {synctex: yes}
% !arara: makeindex
% arara: biber
% arara: pdflatex: {synctex: yes}
% arara: pdflatex: {synctex: yes}

\documentclass[a4paper,twoside,11pt]{reviewresponse}
% 1. Load and set up proper language packages
%\usepackage[utf8x]{inputenc}
\usepackage[utf8]{inputenc}
\usepackage[T1]{fontenc}
\usepackage[english]{babel}
\usepackage{xspace}

\usepackage[defaultlines=4,all]{nowidow}%forbid widow or orphan
% https://tex.stackexchange.com/questions/229638/package-biblatex-warning-babel-polyglossia-detected-but-csquotes-missing
\usepackage{csquotes}

% Hack to try to make acmart work with biblatex: 
%https://tex.stackexchange.com/questions/37076/is-it-possible-to-load-biblatex-with-a-class-that-has-already-loaded-natbib
%\let\citename\relax
\usepackage[
backend=bibtex, % works on Overleaf
% backend=biber, 
%style=numeric-comp, %number the references, compact
style=authoryear, %no indices before names; year after names
%style=alphabetic, %style for my thesis
%
sorting=nyvt, %sort in bibliography by name, year, volume, title
%sortcites, %sort in content by numbering
hyperref=true,%transform citations and back references into 
%clickable hyperlinks
backref=true, %to print back references in the bibliography
backrefstyle=three, %Compress three or more consecutive pages 
%to a range
%,
%citestyle=authoryear,
maxbibnames=9, %the author names appearing in bibliography
maxcitenames=2, %the author names appearing in text
giveninits=true,
uniquename=init, % see tex.stackexchange.com/questions/505654/
%terseinits=true,%remove periods after initials of first names
isbn=false,
%doi=false,url=false,
eprint=false,
%dashed=false,
%useprefix=false,%sort bibliography use prefix like "van"
date=year, %so that the month will not shown with the year
natbib=true,
%style=ACM-Reference-Format,
language=american,
abbreviate=true, dateabbrev=true,
]{biblatex}

% \addbibresource{../Reference/BibReference.bib} % works on local computer
\addbibresource{Reference/BibReference.bib} % works on Overleaf
% \addbibresource{BibReference.bib} % The copy in this folder
\DeclareNameAlias{author}{family-given}%lastname before firstname


%if a reference has doi, then we print only doi in bibliography;
%otherwise, we print only url
%we don't use eprint
%we must deactivate eprint by using "eprint=false"
\renewbibmacro*{doi+eprint+url}{% 
	\iftoggle{bbx:url} 
	{\iffieldundef{doi}{\usebibmacro{url+urldate}}{}} 
	{}% 
	%	\newunit\newblock 
	%	\iftoggle{bbx:eprint} 
	%	{\usebibmacro{eprint}} 
	%	{}% 
	\newunit\newblock 
	\iftoggle{bbx:doi} 
	{\printfield{doi}} 
	{}}	


\newcommand{\Astar}{A$^{\!\star}$\xspace}

% 2. Complete the paper data
\newcommand{\myAuthors}{
{Dongliang Peng, 
Martijn Meijers, 
Peter van Oosterom} %\\ {} other authors
}


\newcommand{\myAuthorsShort}{Dongliang Peng et. al.}
\newcommand{\myEmail}{}
\newcommand{\myTitle}
{Initial responses to the reviewers of paper 
``Generalizing simultaneously to support smooth zooming:
Case study of merging area objects''}
\newcommand{\myShortTitle}{Finding Optimal Sequences for Area Aggregation}
\newcommand{\myJournal}{International Journal of Geographical Information Science}
\newcommand{\myDept}{
    {GIS Technology, Delft University of Technology, The Netherlands}}
%%%%%%%%%%%%%%%%%%%%%%%%%%%%%%%%%%%%%%%%%%%%%%%%%%%%%%%%%%%%%%%%%%%%%%%%%%



%\usepackage[linktoc=all]{hyperref}
\usepackage[linktoc=all,bookmarks,
bookmarksopen=true,bookmarksnumbered=true]{hyperref}

\hypersetup{
	pdfauthor = {\myAuthorsShort},
	pdftitle = {\myTitle},
	pdfsubject = {\myJournal\xspace},
	colorlinks = true,
	linkcolor=black!70!green,          % color of internal links
	citecolor=black!70!green,        % color of links to bibliography
	filecolor=magenta,      % color of file links
	urlcolor=black!70!green           % color of external links
}


\newcommand{\e}[1]{\times 10^{#1}}
\newcommand{\fig}{Figure~}
\newcommand{\eq}{Equation~}
\newcommand{\fo}{Formula~}
\newcommand{\sect}{Section~}
\newcommand{\tab}{Table~}
\newcommand{\chap}{Chapter~}
\newcommand{\figs}{Figures~}
\newcommand{\eqs}{Equations~}
\newcommand{\fos}{Formulas~}
\newcommand{\sects}{Sections~}
\newcommand{\tabs}{Tables~}
\newcommand{\chaps}{Chapters~}
\newcommand{\p}{p.~}
\newcommand{\pp}{pp.~}
\newcommand{\eg}{e.g.,}
\newcommand{\ie}{i.e.,}

\renewcommand{\labelitemi}{\textendash}




\begin{document}
	
\thispagestyle{plain}

\begin{center}
	{\Large\myTitle} \vspace{0.5cm} \\
%	{\large\myJournal} \vspace{0.5cm} \\
	\today \vspace{0.5cm} \\
%	\myAuthors \\
%	\url{\myEmail} \vspace{1cm} \\
%	\myDept
\end{center}

\tableofcontents

\begin{abstract}
We submitted an earlier version of this paper to 
\emph{Journal of Geovisualization and Spatial Analysis}
on September 27, 2021.
According to the concerns of the reviewers,
the Editor in Chief, Prof. Haowen Yan, 
suggested that we should incorporate major revisions.


We thank the anonymous reviewers for their comments.
We have improved our paper according to the comments and
marked most of the changes in red.
In this document, 
we provide detailed responses to each of the reviewers' comments. 
The following list summarizes our most important changes:
\begin{itemize}
    \item We have used more induction texts in our abstract 
    to present the whole picture.
    \item We have moved some discussions to more proper places
    so that the paper is more easily to be understood.
    \item We have improved the description of the greedy algorithm.
    \item We have explained many concepts better in the related work
    (\eg~merging, amalgamation, aggregation, GAP, tGAP, and Eater).
    \item We have explained more about the tables in Appendix A.     
\end{itemize}
\end{abstract}

\section{Reviewer 1}
\setcounter{comments}{-1} %the counter will start from 0

\phantomsection % see https://latex.org/forum/viewtopic.php?t=22814
\rcommentnoskip{
The main idea of this paper (performing simultaneous smooth zooming in the context of SSC) is interesting, and is as far as I know a novel topic. Generally, the paper is good, and therefore my comments are on smaller details.
}
\textbf{Response.} Thanks for the positive comment.


\rcomment{
When starting reading the paper, I missed a statement by the author in which way they wanted to perform the simultaneous generalization. Since working with e.g. SSC prohibit connected objects to be treated, a choice has to be made of how to select the object. But the authors do not really describe their ideal situation. This topic is, however, treated in the later part of the paper (page 11 row 33-44 (right) page 13 row 37-53) where they discuss the shortcoming of the proposed greedy algorithms if a too large value of the simultaneous parameter is used (which is a quite obvious effect). It would be good if the author could bring up this topic early in the paper (and later refer back to their vision when they know the result).
}
\textbf{Response.} 
We have brought the topic to the end of Section 3.1
and referred back to it later.
The text we have added is that
``The ideal situation to apply our method is that 
small areas distribute evenly and the areas do not have holes.
The reason is as follows.
We wish to use a rather large simultaneous parameter~$r_\mathrm{simul}$
so that many events can share their merging time.
However, if the small areas do not distribute evenly,
then some small areas will be blocked and kept 
while some larger areas will be merged,
which is unreasonable.
If an area has many holes, where each hole is filled with an area, 
then each step a hole merging into its surrounding area will forbid
other holes to merge into it.
This situation results in that 
some holes are merged until the scale is very small.
A typical example is that 
a built-up area contains many buildings as holes.''





\rcomment{
The weakest part of the paper is the description of the greedy algorithm (I might have made mistake while reading this part, but my impression is that this description really needs improvements).
}
\textbf{Response.} 
We have significantly improved the description of the greedy algorithm
(see Section 3.1).
We used a dashed rectangle to mark the process of 
finding merging events for a single step.
We made items when describing the greedy algorithm
so that the structure is clearer.
Please also see our responses to other comments in this document.





\rcomment{
While reading it I got a bit uncertain of 
whether the greedy algorithm is in real time or 
if it is used in a preprocess. 
I might have missed something. 
Make sure this is clear.
}
\textbf{Response.} 
The greedy algorithm is used in a preprocess.
We have added that
``Note that the steps of 
Sections 3.1, 3.2, and 3.3
are done in a preprocess, and the final results are saved in files.
Then the files are sent to the client on request
when a user is browsing the map,
where the steps of Section 3.3 and 3.4
are done in real time.''

\rcomment{
State early in the description that 
the greedy algorithm aims at defining which events 
that should be triggered at a certain step. 
This becomes clear in the sentence starting with 
``The merging'' on page 7 row 59, 
but the reader needs this information earlier.
}
\textbf{Response.} 
We have brought the description to 
the beginning of Section 3.1,
where the text is
``We use a greedy algorithm 
to find the simultaneous merging events for all the steps.
The merging events will be stored as records in the tGAP database tables
(see Figure 6).
Some instances of the tables are shown in 
Table 1 of Section 3.2.''



\rcomment{
Page 6 row 43 right. Here you use $M_0$, 
while in start of Figure 6 $M_s$ is used.
}
\textbf{Response.} 
$M_0$ is a case of $M_s$,
where $s=0$.
We have added ``The process starts with a detailed map of area objects.
The map is denoted by $M_s$, where state~$s$ is 0 at this point.''

\rcomment{
Page 6, row 42 right. 
Has the termed ``blocked'' been described earlier?
}
\textbf{Response.}
We used this term earlier without describing it.
We have added 
``In order to realize the requirement,
we \emph{block} the pair of areas of a merging event
as well as their neighbors.
These areas become \emph{blocked areas}.
The areas are \emph{free} 
if they are not blocked yet.''

\rcomment{
Equation 1: You have to explain the symbols used 
refer to the ceiling function in the where-sentence after the equation.
}
\textbf{Response.}
We have added that
expression $\lceil x \rceil$ returns the ceiling of $x$,
which is the smallest integer greater than or equal to $x$.

\rcomment{
Page 6 row 47 right. Is ``free area'' defined?
}
\textbf{Response.}
We have added the descriptions of 
\emph{blocked areas} and \emph{free areas}
in the paragraph before Section 3.1
(also see our response to Reviewer 1 Comment 6).

\rcomment{
Figure 5. In the caption you write that this is a process of 
``merging events for a step''. 
In the bottom of figure 6 it is stated ``all the step''.
}
\textbf{Response.}
The statements are correct.
In the flowchart,
We have marked the process of finding merging events for a single step.


\rcomment{
Figure 6.  Should you really state 
the value of the simultaneous parameter here? 
The greedy algorithms is not dependent on the value setting.
}
\textbf{Response.}
The simultaneous parameter has influence on 
how many merging events will be found in a single step.
However, the parameter is not the focus of the figure,
so we have removed the value of the simultaneous parameter.


\rcomment{
Figure 1, page 3 row 43--48: 
A more detailed description of the ``Eater'' is required. 
It is not easy to understand what ``Eater'' is 
in the context of the graphic figure.
}
\textbf{Response.}
We have added more details to explain ``Eater''
(see Section 2.2).
In detail, we have added
``First, the interior of the loser is triangulated
with a Delaunay triangulation
(see Figure 2a).
Second, the triangles are visited starting from the boundary
between the winner and the loser:
If there are triangles with two shared edges, 
then the visiting starts from the shared vertex of the two edges;
otherwise, it starts from the shared edges.
During visiting, the vertices of the triangles 
are assigned with increasing $z$-values,
and the tilted triangles are generated,
which become the boundaries of polyhedra in the SSC.
When slicing the SSC with a horizontal plane,
the eating process is presented as shown in Figure 2b.''

\rcomment{
Figure 3: Would recommend that
 three separate lists are used for the three workflow. 
 That is, instead of 3A-O it should be 3A1-3A3, 3B1-3B7 and 3C1-3C5.
  You also need to comment the figure p, q and r in the figure caption. 
  When reading the text here it is not known what these figures refer to. 
  This become not clear until a few pages later.
}
\textbf{Response.}
We have relabeled the subfigures according to the recommendation. 
We have commented the extra subfigures in the caption,
which is
``Subfigures~(a4), (b8), (c6), and (c7) 
are the map representations during the scale transitions,
where their corresponding states are indicated by the gray dots.''



\rcomment{
Page 8, row 39-42. Make a short note that 
some of the attributes are inherited 
(from the tGAPTopolObject superclass).
}
\textbf{Response.}
We have added in Section 3.2 that
``Note that both class Face and class Edge inherit
 the attributes from superclass tGAPTopolObject.''

\rcomment{
Page 8 row 42: 
Here you should comment the relationship
 between ``$s_\mathrm{low}$'' - ``$s_\mathrm{high}$'' 
 with the s value used in the flowchart in Figure 5.
}
\textbf{Response.}
We have added that
``The values of $s_\mathrm{low}$ and $s_\mathrm{high}$
are assigned when all the pairs of areas are merged
(see the step in Figure 7).
In detail, all those pairs of areas have 
$s_\mathrm{high} = s + n_\mathrm{event}$,
and the generated areas will have 
$s_\mathrm{low} = s + n_\mathrm{event}$.''

\rcomment{
Equation 2: Both $s$ and ``$s_\mathrm{high}$'' are input to the function. 
Not only ``$s_\mathrm{high}$''.
}
\textbf{Response.}
Both ``$s_\mathrm{high}$'' and ``$S_\mathrm{high}$'' are inputs.
Variable $s$ represents each of $S_\mathrm{high}$'s elements.
We have added $S_\mathrm{high}$ as an input of Equation 2.


\rcomment{
Table 1b: Is the value of fid 10 really correct? 
Could "smerge" be lower than "sslow"
}
\textbf{Response.}
No, the value was wrong. We have fixed the problem.

\rcomment{
Page 9, row 26 left (right). "$S_\mathrm{high}$" is a vector. 
Normally a vector is written in bold (and not italics)
}
\textbf{Response.}
We have changed the notation. 
Now, we use $\mathrm{\textbf{S}_{high}}$.

\rcomment{
Figure 8. Is there any reference in the text to this figure?
}
\textbf{Response.}
Yes, the reference is in Section 3.5.

\rcomment{
Figure 10a: Do not get which are the overview map is of.
}
\textbf{Response.}
We have marked the part in Figure 10b
(see the revised version).

\rcomment{
Page 12, row 56--60. Be more explicit that 
the sequential and simultaneous approach is compared here.
}
\textbf{Response.}
We have rephrased the beginning of the paragraph.
Now, we have
``Comparing to the map based on the single merging,
the map based on the simultaneous merging indeed 
provides smoother zooming.
We set zooming factor~$f_\mathrm{zoom}=1$ and 
zooming duration~$t_\mathrm{zoom}=1 s$ 
(see Section 3.5).
The map based on the single merging gives the impression
of discrete scale transition, 
where it is difficult to see a winner expands over a loser.\footnote{%
See the web map at
\url{https://congengis.github.io/webmaps/2021/10/merge/limburg-single-merging}.} 
The reason is that the merging happens too fast,
so the time for animation available is too short.
This is also the case when 
we use simultaneous parameter~$r_\mathrm{simul}= 0.001$.''


\rcomment{
Figure 11. For me this figure is not that interesting. The main contribution of the simultaneous approach is how the dynamics is improved. A figure that shows the difference at a certain point in time is not that interesting in my view (if I interpret this figure right). Consider to remove this figure.
}
\textbf{Response.}
We replaced the figure with a figure to expose a problem
when simultaneous parameter $r_\mathrm{simul}= 0.1$.
We also added some text to describe the problem,
where the description is
``Figure 12 shows a problem 
when we use simultaneous parameter $r_\mathrm{simul}= 0.1$.
That is, some tiny and relatively unimportant areas stay 
until the scale is quite small,
where they should be merged when the scale is larger.
This problem has been mentioned in Section 3.1.''

\rcomment{
Page 13, row 41, right column. There is a grammatical error here "objects, We"
}
\textbf{Response.}
We have fixed this grammatical error.

\rcomment{
Page 14 row 27-28. It is not easy to understand how the mthod of Ai would solve this issue.
}
\textbf{Response.}
After a careful consideration, 
we don't think that the small parcels should grow and merge each other.
If they do grow and merge each other, 
they may fill the place of the background area,
which is unreasonable.
So, we have removed the paragraph. 


\rcomment{
Page 14, row 10-14 right column. Here you bring up label placement on a few row. If you include labels you need to explain more about how labels should act in the context of smooth generalization. And also how the two references for label placement are connected to this approach. For me this part feels very uncertain and I would recommend the authors to remove these sentences.
}
\textbf{Response.}
Your suggestion makes sense.
We have removed the discussion about label placement.

\rcomment{
Page 14. Row 18-25 right column. 
Here the authors mention the approach of whether the foreground or background 
should be visible for zooming in and out. 
This is of course a topic, but hardly the main topic 
when it comes to integrating thematic information to the SSC. 
The harmonization of scale levels in the foreground and background 
(and object relationships) 
is a much more tricky problem that needs to be solved 
if the SSC can be used as a base map for thematic information.
}
\textbf{Response.}
We have removed the discussion about the foreground and the background.

\rcomment{
Reference. Some references starting with S and T are not in alphabetic order.
}
\textbf{Response.}
This is because we used  ``\textbackslash v\{S\}uba''
instead of ``\{\textbackslash v S\}uba'' in the .bib file.
We have fixed the problem.

\rcomment{
Appendix A. The tables need to more explained. Otherwise it is very hard to understand the SQL statements.
}
\textbf{Response.}
We have added in the first paragraph of Appendix A that
``In table \emph{class\_weights}, 
field code stores the codes of the classes,
and field weight stores the class weight.
Table \emph{class\_comp\_matrix} 
stores the distances and the compatibility values between the classes.
The distance is defined based on a tree similar to Figure 3.
The compatibility value is between 0 and 1.
If two areas are with the same class, then the compatibility value is 1.''

\rcomment{
Appendix. The numbering of equation should not be 8 etc. Rater B1 etc.
}
\textbf{Response.}
We have renumbered the equations according to the suggestion.

\clearpage

\section{Reviewer 2}
\setcounter{comments}{-1} %the counter will start from 0

\rcommentnoskip{
The manuscript presented a simultaneous generalization method with progressive transformation effects. Using the example of land-use parcel merging explains the idea of progressive transformation process. The whole manuscript discussed a complete process including the basic idea, the algorithm process, the attached data and web coding source. The presented method is an innovation strategy in map generalization to reflect the detailed scale transformation and other temporal change. The following comments can be referenced.
}
\textbf{Response.} 
Thanks for the summary!

\rcomment{
The abstract writing needs to use more induction texts to summarize the main ideas of this method from a high level perspective, rather than detailed process description. The advantages and applications of this method needs to evaluate.
}
\textbf{Response.}
We have added more induction text in the abstract,
where we also presented the main advantage,
which is to help map users better keep track of 
their interested objects during zooming.
We have also removed some sentences of the abstract.
The added induction text is
``the map should change correspondingly to present
geographical information at proper levels.
A way to help map users better keep track of their interested objects
is to change the map smoothly.
This paper focuses on the problem of providing smooth changes of area objects.''


\rcomment{
The gradual operation of map generalization is the main idea in the manuscript. In the introduction it needs to discuss this question deeply. Jump from the concrete merging operation and from a common perspective discuss the concept gradual generalization. The characteristics, advantages, applications of this new method is expected to use more texts to describe. Suggest consider the generalization of temporal evolution phenomena, such as soil, flooding region.
}
\textbf{Response.}
We have discussed more deeply in the introduction.
The added text is
``The scale transition is realized by switching 
between the maps at corresponding scales. 
This strategy often brings large and discrete changes, 
which confuse users.
In order to provide users with better experience of zooming,
we propose to realize the scale transition with smooth changes.
In detail, each object on the map should be changed smoothly
when the scale changes.
For example, a smooth way to simplify a polyline is 
to move some of the vertices to a straight line,
and a smooth way to make a polygon disappear is to fade it out.
Because all the objects are changed smoothly,
users can more easily keep track of their interested objects.
The technology to realize the smooth scale transition is known as
\emph{continuous map generalization (CMG)}.''
and
``Further, CMG can be also used 
to simulate some temporal evolution of phenomena,
for example, how a flooding submerges land.''

\rcomment{
The progressive transformation in web generalization 
can be referenced citing some related papers.
}
\textbf{Response.}
We have added Section 2.5:
Gradual transformation in web environment,
where we reviewed two more papers, which are
\citet{Huang2016Webmap} and \citet{Peng2020Viewer}

\rcomment{
As for the polygon generalization, in section 2 you mentioned aggregation, merging and others. The difference of aggregation, amalgamation and merging needs to distinguish.
}
\textbf{Response.}
We have explained the differences of the three operators
in the beginning of Section 2,
where the the text is
``Merging, amalgamation, and aggregation  
are three popular operators of combining area objects. 
According to \citet{Shea1989Digital},
merging combines neighbor objects, 
which (visually) share their boundaries, into a single one,
and the result has the same dimension as the merged objects.
Amalgamation is different from merging in that 
it combines nearby objects into a single one.
In contrast, aggregation often involves the change of dimension.
For example, points are aggregated into an area.''

\rcomment{
In the section of methodology, 
reference TGAP, SSC by citation and miss some contents. 
Suggest describe these two models using detailed texts since
 they are important support in the manuscript.
}
\textbf{Response.}
We have added detailed text to explain the GAP and the tGAP 
in Section 2.2.
To explain the GAP, we have added
``Each leaf node of the tree represents an area on the map.
Then the least important area is found and 
merged into its most compatible neighbor.
Correspondingly, the former's node 
becomes a child of the latter's node.
This process repeats until the root node is found.
As a result, the GAP tree has been built.
Each level of the GAP tree corresponds to a scale of the map.
As the changes between two neighbor levels of the tree are small,
it is possible to realize scale transitions with small changes.''
To explain the tGAP, we have added 
``The topological GAP (tGAP) structure 
consists of a face tree and an edge tree \citep{vanOosterom2005}.
The aim of proposing the tGAP structure is to minimize the redundancy,
where each vertex is recorded once while may be presented in many scales.''
There is already a detailed description of the SSC in the same section.

\rcomment{
The figure 4 needs to explain in detail. 
What is the difference between left and right? 
How to reflect the merging event in SSC?
}
\textbf{Response.}
We have added more details in the section of methodology,
where the text is
``The differences of the two SSCs result in 
the two different merging processes.
For example, at $z=50$ of Figure 5a, 
there is only one polyhedron with the tilted face,
hence there is only a pair of polygons merging in Figure 4b8.
In comparison, there are two polyhedra
with tilted faces at $z=50$ of Figure 5b,
and there are two pairs of polygons merging in Figure 4c6.''

\rcomment{
As for the gradual merging operation, you mentioned several times in the manuscript. But how to realize in geometric process , how to triangulate and how to use triangle cell to gradually eating the neighbor, needs to use a proper figure to illustrate these processes. This question plays an important role in the manuscript.
}
\textbf{Response.}
We have added more details to explain ``Eater''
in Section 2.2, and we have included a figure from
\citet{Meijers2020Web}.
The text added is 
``First, the interior of the loser is triangulated
with a Delaunay triangulation
(see Figure 2a).
Second, the triangles are visited starting from the boundary
between the winner and the loser:
If there are triangles with two shared edges, 
then the visiting starts from the shared vertex of the two edges;
otherwise, it starts from the shared edges.
During visiting, the vertices of the triangles 
are assigned with increasing $z$-values,
and the tilted triangles are generated,
which become the boundaries of polyhedra in the SSC.
When slicing the SSC with a horizontal plane,
the eating process is presented as shown in Figure 2b.''


\rcomment{
How to define the granularity in the gradual merging if considering the scale change range? The change granularity for the merging From 1:10K to 1:50K and 1:10K to 1:100K should be different, I think. Considers the time costs, a large scale change range needs to more time and correspondingly coarse details.
}
\textbf{Response.}
The amount of scale change is based on the zooming factor
(see Equation C1 in Appendix C).
The scale change influences the number of events
(see Equation C2 in Appendix C).
According to the number of events, 
we can compute the number of steps.
Finally, all the steps in a zooming duration 
have the same merging time.
We have added this explanation into Appendix C.



\rcomment{
Based on GAP model, the minor areas need to arrange in a linear sequence according to the importance. If some are in the similar importance level, and may be neighbor to each other, how to process in your block and eating strategy?
}
\textbf{Response.}
In this case, only one of the minor areas is allowed to merged in a step.
We have added an explanation of this case at the end of Section 3.1
(also see our response to Reviewer 1 Comment 1 in this document).
The problem does happen in our case study 
when we use simultaneous parameter $r_\mathrm{simul}=0.1$.









\printbibliography
	
\end{document}

