% 1. Load and set up proper language packages
%\usepackage[utf8x]{inputenc}
\usepackage[utf8]{inputenc}
\usepackage[T1]{fontenc}
\usepackage[english]{babel}
\usepackage{xspace}

\usepackage[defaultlines=4,all]{nowidow}%forbid widow or orphan
% https://tex.stackexchange.com/questions/229638/package-biblatex-warning-babel-polyglossia-detected-but-csquotes-missing
\usepackage{csquotes}

% Hack to try to make acmart work with biblatex: 
%https://tex.stackexchange.com/questions/37076/is-it-possible-to-load-biblatex-with-a-class-that-has-already-loaded-natbib
%\let\citename\relax
\usepackage[
% backend=bibtex,
backend=biber, 
%style=numeric-comp, %number the references, compact
style=authoryear, %no indices before names; year after names
%style=alphabetic, %style for my thesis
%
sorting=nyvt, %sort in bibliography by name, year, volume, title
%sortcites, %sort in content by numbering
hyperref=true,%transform citations and back references into 
%clickable hyperlinks
backref=true, %to print back references in the bibliography
backrefstyle=three, %Compress three or more consecutive pages 
%to a range
%,
%citestyle=authoryear,
maxbibnames=9, %the author names appearing in bibliography
maxcitenames=2, %the author names appearing in text
giveninits=true,
uniquename=init, % see tex.stackexchange.com/questions/505654/
%terseinits=true,%remove periods after initials of first names
isbn=false,
%doi=false,url=false,
eprint=false,
%dashed=false,
%useprefix=false,%sort bibliography use prefix like "van"
date=year, %so that the month will not shown with the year
natbib=true,
%style=ACM-Reference-Format,
language=american,
abbreviate=true, dateabbrev=true,
]{biblatex}

%\addbibresource{../Reference/BibReference.bib} % works on local computer
\addbibresource{Reference/BibReference.bib} % works on Overleaf
% \addbibresource{BibReference.bib} % The copy in this folder
\DeclareNameAlias{author}{family-given}%lastname before firstname


%if a reference has doi, then we print only doi in bibliography;
%otherwise, we print only url
%we don't use eprint
%we must deactivate eprint by using "eprint=false"
\renewbibmacro*{doi+eprint+url}{% 
	\iftoggle{bbx:url} 
	{\iffieldundef{doi}{\usebibmacro{url+urldate}}{}} 
	{}% 
	%	\newunit\newblock 
	%	\iftoggle{bbx:eprint} 
	%	{\usebibmacro{eprint}} 
	%	{}% 
	\newunit\newblock 
	\iftoggle{bbx:doi} 
	{\printfield{doi}} 
	{}}	


\newcommand{\Astar}{A$^{\!\star}$\xspace}

% 2. Complete the paper data
\newcommand{\myAuthors}{
{Dongliang Peng, 
Martijn Meijers, 
Peter van Oosterom} %\\ {} other authors
}


\newcommand{\myAuthorsShort}{Dongliang Peng et. al.}
\newcommand{\myEmail}{}
\newcommand{\myTitle}
{Initial responses to the reviewers of paper 
``Generalizing simultaneously to support smooth zooming:
Case study of merging area objects''}
\newcommand{\myShortTitle}{Finding Optimal Sequences for Area Aggregation}
\newcommand{\myJournal}{International Journal of Geographical Information Science}
\newcommand{\myDept}{
    {GIS Technology, Delft University of Technology, The Netherlands}}
%%%%%%%%%%%%%%%%%%%%%%%%%%%%%%%%%%%%%%%%%%%%%%%%%%%%%%%%%%%%%%%%%%%%%%%%%%



%\usepackage[linktoc=all]{hyperref}
\usepackage[linktoc=all,bookmarks,
bookmarksopen=true,bookmarksnumbered=true]{hyperref}

\hypersetup{
	pdfauthor = {\myAuthorsShort},
	pdftitle = {\myTitle},
	pdfsubject = {\myJournal\xspace},
	colorlinks = true,
	linkcolor=black!70!green,          % color of internal links
	citecolor=black!70!green,        % color of links to bibliography
	filecolor=magenta,      % color of file links
	urlcolor=black!70!green           % color of external links
}

\newcommand{\noticecomment}[1]{\textcolor{red}{#1}}
\newcommand{\e}[1]{\times 10^{#1}}
\newcommand{\fig}{Figure~}
\newcommand{\eq}{Equation~}
\newcommand{\fo}{Formula~}
\newcommand{\sect}{Section~}
\newcommand{\tab}{Table~}
\newcommand{\chap}{Chapter~}
\newcommand{\figs}{Figures~}
\newcommand{\eqs}{Equations~}
\newcommand{\fos}{Formulas~}
\newcommand{\sects}{Sections~}
\newcommand{\tabs}{Tables~}
\newcommand{\chaps}{Chapters~}
\newcommand{\p}{p.~}
\newcommand{\pp}{pp.~}
\newcommand{\eg}{e.g.,}
\newcommand{\ie}{i.e.,}

\renewcommand{\labelitemi}{\textendash}

